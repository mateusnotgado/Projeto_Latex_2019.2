 \documentclass{article}
\usepackage[utf8]{inputenc}

\title{FI582 -FISICA PARA COMPUTACAO}
\author{mcbt}
\date{November 2019}
\begin{document}

\maketitle

\section{Introducão}
Física para Computação é uma disciplina que visa desenvolver no aluno os conceitos básicos de física.Mais especificamente de Mecânica,Termodinâmica e Elétrica.~\cite{garcia2012physics}.
\subsection{Requerimentos}
A cadeira requer apenas uma base de Cálculo Diferencial e Integral.E vale ressaltar que se o estudante tiver  uma boa base de física do Ensino Médio a cadeira se tornará algo bem trivial ,uma vez que,os assuntos são praticamente os mesmos dos vistos no Ensino Médio.
\section{Relevância}
Essa cadeira é importante, pois, tendo o aluno uma base em física ele terá muito mais possibilidades de mercado podendo trabalhar em áreas como : engenharia civil, engenharia elétrica, programação de jogos,programação de foguetes, entre muitas outras.Uma vez que, os programas de computadores se fazem muito importantes atualmente para todas as profissões relacionadas a física ,já que, eles se encarregam de fazer todo o "trabalho braçal" que muitas vezes seria impossivel de fazer a mão como ,por exemplo,os cálculos metereológicos.~\cite{pang1999introduction}\\
~\cite{moody2009physics}
Além do mais, os conhecimentos de física se fazem importante para entender plenamente como funciona a parte de Hardware do computador.E assim, obtendo um maior entendimento sobre o funcionamento de um computador e suas propiedades, podendo assim ,por exemplo, diagnotiscar com maior facilidade quaisquer problemas que um computador possa vir a ter ou então fazer com muito mais propiedade a ponte entre Hardware e Software.
\section{Relação com outras diciplinas }
Como foi citado, brevemente, acima a física é fundamental para entender a infraestrutura de Hardware.Todos os avanços tecnológicos computacionais da atualidade ,assim como os futuros, só foram póssiveis pelo avanço da física,especialmente da física quântica que permitiu que os computadores ficassem compactos e adquirissem o formato que hoje tem.\\

Além do mais, a física é importante para todas as cadeiras que envolvem física(meio óbvio), tais como :róbotica, como já citado infraestrutura de Hardware,história e futuro da computação (uma vez que está ligada a inovação tecnológica que por sua vez está ligada ao desenvolvimento da ciência),entre outros.
\bibliographystyle{plain}
\bibliography{mcbt.bib}
\end{document}
